\section{Introducción}

Por la creciente importancia que la tecnología digital tiene hoy en día y gracias a su transversalidad en la vida cotidiana, las TIC se han vuelto una parte integral del proceso educativo. El uso de dispositivos electrónicos en el aula es una realidad y por consiguiente, la implementación y el desarrollo de propuestas educativas que permitan la evolución cognitiva de los discentes y un aprovechamiento de las nuevas tecnologías se vuelven de especial interés por sus ventajas como recurso educativo. En consecuencia, y como dice \citet{chavarria_software_2011}, la discusión sobre el uso de software y hardware libre como herramienta educativa, se vuelve trascendental para lograr el desarrollo de conceptos como la soberanía tecnológica dentro del aula.

La robótica utilizada en el  contexto educativo tiene múltiples  aplicaciones para ser aprovechada como herramienta transversal para el aprendizaje de conceptos de matemática, física y lenguajes de programación. Tomando al construccionismo de \citet{papert_childrens_1993} como teoría pedagógica central, podemos decir que la robótica es una herramienta construccionista, la cual permite a los discentes, abordar  la construcción de su propio conocimiento en base de diseñar y fabricar un robot de carácter pedagógico.  

Su carácter multidisciplinario permite abordar distintas etapas de la construcción del conocimiento por parte de los discentes, ademas de poder ser aplicada en una gran variedad temáticas, y su “espectacularidad” (entendida como la capacidad de generar algún tipo de asombro en la población estudiantil), ayuda a los docentes en la tarea de impartir la curricula planteada en el curso.

Sin embargo, la gran dificultad técnica y la cantidad de conocimientos específicos necesarios para desarrollar un robot (como electrónica, mecánica y ciencias computacionales) , ademas de un costo generalmente elevado de los kits que se consiguen en el mercado, hace que la enseñanza de robótica en las escuelas, este relegada generalmente a colegios que pueden financiar los costes de capacitación y adquisición de estos kits comerciales.

El proyecto ICARO busca desarrollar una solución técnica basada en hardware de especificaciones abiertas y software libre para facilitar la labor del docente a la hora de trabajar los contenidos técnicos complejos que implica el abordaje de una disciplina como la robótica, y ademas abaratar costos, al ser ICARO un proyecto pensado para poder implementarse en pequeña escala, sin equipamiento industrial y con la idea de que los docentes y discentes desarrollen (pensando la producción como el hecho de soldar los componentes electrónicos ) el hardware ICARO, poniendo un fuerte énfasis en la fabricación como una herramienta construccionista y de aprehensión, entendiendo esta practica como una ''herramienta de liberación'' en el sentido planteado por Paulo \citet{freire_pedagogioprimido._2015}.

Por consiguiente, La presente tesis propone un proyecto de transferencia educativo-tecnológico, basado en el diseño y desarrollo de hardware electrónico para el desarrollo de elementos de robótica, domotica o automatización, y software de control para dicho hardware, basado en los principios de soberanía tecnológica y las licencias de software libre (versión GPLV3).
