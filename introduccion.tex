\section{Introducción}

Dada la creciente importancia que la tecnología digital tiene hoy en día y gracias a su transversalidad en la vida cotidiana, las TIC se han vuelto una parte integral del proceso educativo. Por consiguiente, la implementación y el desarrollo de propuestas educativas que permitan la evolución cognitiva de los discentes y el aprovechamiento de las nuevas tecnologías se vuelven de especial interés por sus ventajas como recurso educativo. En consecuencia, y como sostiene \citet{chavarria_software_2011}, la discusión sobre el uso de software y hardware libre como herramienta educativa, cobra especial relevancia para el desarrollo de conceptos como la soberanía tecnológica dentro del aula.

La robótica utilizada en el  contexto educativo tiene múltiples  aplicaciones como herramienta transversal en el aprendizaje de conceptos de matemática, física y lenguajes de programación. Tomando al construccionismo de \citet{papert_childrens_1993} como teoría pedagógica central, podemos decir que la robótica es una herramienta construccionista \citep{pitti_experiencias_2010}, que permite a los discentes abordar  la construcción de su propio conocimiento a partir del diseño y fabricación de un robot de carácter pedagógico.  

Su carácter multidisciplinario permite abordar distintas etapas de la construcción del conocimiento por parte de los discentes. Además, la robótica puede ser, aplicada en una gran variedad de temáticas; su ''espectacularidad'' (entendida como la capacidad de generar algún tipo de asombro en la población estudiantil), ayuda a los docentes en la tarea de impartir la curricula planteada en el curso. Sin embargo, la dificultad técnica y la cantidad de conocimientos específicos necesarios para desarrollar un robot - electrónica, mecánica y ciencias computacionales-, además del costo generalmente elevado de los kits que se consiguen en el mercado, hace que la enseñanza de robótica en las escuelas quede relegada, generalmente, a colegios que pueden financiar los costes de capacitación y adquisición de esos kits comerciales.

A partir de lo señalado, el presente trabajo final de licenciatura propone un proyecto de transferencia educativo-tecnológico, basado en el diseño e implementación de hardware electrónico para el desarrollo de elementos de robótica, domótica o automatización, y software de control para dicho hardware, basado en los principios de soberanía tecnológica y las licencias de software libre (versión GPLV3). El objetivo general es transferir el conocimiento técnico para independizar a los colegios de proveedores y fabricantes de hardware especializado en robótica educativa, preparando a los docentes y discentes en la fabricación y uso/implementación del hardware ICARO NP07 en los respectivos espacios curriculares. Dicho hardware es una herramienta pedagógica diseñada para la enseñanza de lenguajes de programación. 

El proyecto ICARO busca desarrollar una solución técnica basada en hardware de especificaciones abiertas y software libre, para facilitar la labor del docente a la hora de trabajar los contenidos técnicos complejos que implica el abordaje de una disciplina como la robótica. Además, permite abaratar costos, al ser un proyecto pensado para implementarse en pequeña escala, sin equipamiento industrial y con la idea de que los docentes y discentes desarrollen (entendiendo la producción como el hecho de soldar los componentes electrónicos) el hardware ICARO. El énfasis esta puesto aquí en (el acto de) la fabricación como una herramienta construccionista y de aprehensión, entendiendo esta práctica como una ''herramienta de liberación'' en el sentido planteado por Paulo \citet{freire_pedagogioprimido._2015}.

El proyecto de transferencia tecnológica educativa \citep{bareno2011metodologia} estará dividido en seis etapas, a lo largo de las cuales se investigará para determinar la viabilidad técnica - presupuestaria de la implementación de hardware libre en la enseñanza escolar. El proyecto contempla la capacitación de docentes y discentes para la fabricación, uso y mantenimiento del hardware propuesto. A partir del proceso de ''auto fabricación'' de las placas ICARO NP07,  docentes y discentes de la institución se apropiaran de los conceptos técnicos necesarios para poder mantener y reparar el hardware y  utilizarlo como una herramienta educativa, principalmente por sus posibilidades la enseñanza de lenguajes de programación. Al respecto, vale recordar que la ''programación'' es considerada como una ''capacidad necesaria'' segun la ley Nacional de Educación N\grad  26.206.
%
