
%%%
% Plantilla de Libro
% Modificación de una plantilla de Latex de Mathias Legrand (legrand.mathias@gmail.com)
% sobre modificaciones de Vel (vel@latextemplates.com) para adaptarla 
% al castellano y a las necesidades de escribir informática y matemáticas.
%
% Modificada por: Mario Román
%
% License:
% CC BY-NC-SA 3.0 (http://creativecommons.org/licenses/by-nc-sa/3.0/)
%%%

%%%%%%%%%%%%%%%%%%%%%%%%%%%%%%%%%%%%%%%%%
% The Legrand Orange Book
% LaTeX Template
% Version 2.0 (9/2/15)
%
% This template has been downloaded from:
% http://www.LaTeXTemplates.com
%
% Mathias Legrand (legrand.mathias@gmail.com) with modifications by:
% Vel (vel@latextemplates.com)
%
% License:
% CC BY-NC-SA 3.0 (http://creativecommons.org/licenses/by-nc-sa/3.0/)
%
% Compiling this template:
% This template uses biber for its bibliography and makeindex for its index.
% When you first open the template, compile it from the command line with the 
% commands below to make sure your LaTeX distribution is configured correctly:
%
% 1) pdflatex main
% 2) makeindex main.idx -s StyleInd.ist
% 3) biber main
% 4) pdflatex main x 2
%
% After this, when you wish to update the bibliography/index use the appropriate
% command above and make sure to compile with pdflatex several times 
% afterwards to propagate your changes to the document.
%
% This template also uses a number of packages which may need to be
% updated to the newest versions for the template to compile. It is strongly
% recommended you update your LaTeX distribution if you have any
% compilation errors.
%
% Important note:
% Chapter heading images should have a 2:1 width:height ratio,
% e.g. 920px width and 460px height.
%
%%%%%%%%%%%%%%%%%%%%%%%%%%%%%%%%%%%%%%%%%

%----------------------------------------------------------------------------------------
%	PACKAGES AND OTHER DOCUMENT CONFIGURATIONS
%----------------------------------------------------------------------------------------

%%% Configuración del papel.
% fleqn: Alinea ecuaciones a la izquierda
%\documentclass[11pt, fleqn, spanish]{book}
\documentclass[a4paper,openright,openany,12pt]{book}
%%% Castellano.
% noquoting: Permite uso de comillas no españolas.
% lcroman: Permite la enumeración con numerales romanos en minúscula.
% fontenc: Usa la fuente completa para que pueda copiarse correctamente del pdf.
%\usepackage[spanish,es-noquoting,es-lcroman]{babel}
%\usepackage[utf8]{inputenc}
%\usepackage[T1]{fontenc}
%\selectlanguage{spanish}
\usepackage{setspace}
\usepackage{float}


\usepackage[T1]{fontenc} % con esto me reconoce los acentos
\usepackage[spanish]{babel}
\usepackage[utf8]{inputenc}
\usepackage[hidelinks]{hyperref}
\usepackage{chapterbib}
\usepackage{graphicx} % graficos
\usepackage{apacite}
\usepackage{natbib}
\usepackage{wrapfig}
\usepackage{listings} % para agregar coloreado al codigo fuente

\newcommand{\grad}{$^{\circ} $}
%----------------------------------------------------------------------------------------

\input{structure} % Insert the commands.tex file which contains the majority of the structure behind the template

\spacing{2} % espaciado

\usepackage[
            type={CC},
            modifier={by-sa},
            version={4.0},
            ]{doclicense}

\begin{document}


%----------------------------------------------------------------------------------------
%	TÍTULO
%----------------------------------------------------------------------------------------

\begingroup
\thispagestyle{empty}
\begin{tikzpicture}[remember picture,overlay]
\coordinate [below=12cm] (midpoint) at (current page.north);
\node at (current page.north west)
{\begin{tikzpicture}[remember picture,overlay]
\node[anchor=north west,inner sep=0pt] at (0,0) {\includegraphics[width=\paperwidth]{background}}; % Background image
\end{tikzpicture}};
\end{tikzpicture}
\vfill
\endgroup
%\begin{titlepage}

\begin{center}
\vspace*{-1in}
\begin{figure}[htb]
\begin{center}
\includegraphics[width=8cm]{./figuras/utn_logo.jpg}
\end{center}
\end{figure}

Universidad Tecnologica Nacional\\
\vspace*{0.15in}
Escuela de Posgrado \\
\vspace*{0.6in}
%\begin{large}
Trabajo final de la carrera de licenciatura en tecnología educativa.\\
%\end{large}
\vspace*{0.2in}
\begin{Large}
\textbf{Proyecto de transferencia de tecnología educativa.}
%\textbf{Robótica diseñada con software y hardware libre como recurso educativo para la enseñanza de lenguajes formales.} \\
\end{Large}
\vspace*{0.3in}
\begin{large}

Implementar robótica diseñada con software y hardware libre como recurso educativo para la enseñanza de lenguajes formales.

%A Thesis submitted by Amy Wong for the degree of Doctor of Philosophy in the Mars University\\
\end{large}
\vspace*{0.3in}
\rule{80mm}{0.1mm}\\
\vspace*{0.1in}
\begin{large}
%Supervised by: \\
%Hubert J. Farnsworth \\
Valentin Basel.
\end{large}
\end{center}

\end{titlepage}
% pagina nueva (que no esta numerada)
\newpage
$\ $
\thispagestyle{empty} % para que no se numere esta pagina

%----------------------------------------------------------------------------------------
%	COPYRIGHT
%----------------------------------------------------------------------------------------
\newpage
~\vfill
\thispagestyle{empty}
\noindent \doclicenseThis
\newpage

% Dedicatoria
%\chapter*{}
%\pagenumbering{Roman} % para comenzar la numeracion de paginas en numeros romanos
\thispagestyle{empty}
\begin{flushright}
Programa, o seras programado.

\texttt{Douglas Rushkoff.}


\end{flushright}


%----------------------------------------------------------------------------------------
%	CONTENIDOS
%----------------------------------------------------------------------------------------

\chapterimage{chapter_head_1.pdf} % Table of contents heading image
\pagestyle{empty} % No headers
%\tableofcontents % Print the table of contents itself
\cleardoublepage % Forces the first chapter to start on an odd page so it's on the right
\pagestyle{fancy} % Print headers again
%\newpage
%----------------------------------------------------------------------------------------
%	PART
%----------------------------------------------------------------------------------------
\chapter*{Planificación de clase Numero: 1}
\subsubsection{Temática}
Introducción a la robótica, armado modulo 1 y 2 placa ICARO

\subsubsection{Tiempo}
3 Horas

\subsubsection{Resumen}

La actividad se basa en dar una breve introducción al mundo de los sistemas de automatización y robótica, presentar el proyecto ICARO, reconocer los distinto componentes electrónicos que se usaran durante el taller y comenzar con el ensamblado del modulo 1 y 2 de la guiá de construcción del hardware icaro.

\subsubsection{Objetivos}
que los participantes logren:
\begin{itemize}
  \item Identificar los distintos tipos de robots y sus aplicaciones.
  \item Reconocer los conceptos de sensores y actuadores en un robot o sistema de automatización.
  \item Iniciar a los participantes en Las propiedades del proyecto ICARO e identificar los distintos componentes del mismo.
  \item Identificar componentes electrónicos necesarios para montar el modulo 1 y 2 del hardware ICARO
  \item Desarrollar habilidades para la manipulación de las herramientas necesarias para el ensamblado del hardware ICARO (lápiz soldador de estaño, pinzas, cutter).
  \end{itemize}

\subsubsection{actividades}
inicio:
\begin{itemize}
  \item Introducción al mundo de los sistemas de automatización y robótica mediante un presentación y actividad indagatoria.
  \item Entrega de los materiales necesarios para el armado del modulo 1 y 2 del hardware ICARO.
  \item comentarios para el uso seguro del soldador de estaño.
\end{itemize}
desarrollo:
\begin{itemize}
  \item Prueba de soldadura con estaño.
  \item Identificación de los componentes.
  \item Soldar los componentes del modulo 1 y 2.
  \item Cargar el bootloader en el micro controlador 18f4550.
  \item Probar conexión con la computadora (comandos lsub y dmesg).
  \item Cargar firmware ''ejemplo\_01.icr'' del software icaro\_bloques.
  \item Medir con el multimetro (punta logica) si hay señal en el puerto B (salidas UNL2803).

\end{itemize}

\subsubsection{metodologia}

Luego de la introducción y de repartir los competentes electrónicos, se explica a los participantes como usar el soldador de estaño con recomendaciones para evitar accidentes, después se procede a armar el modulo 1 y 2 del Anexo 2 para armado de las placas np07 de ICARO, los módulos 1 y 2 del Anexo 2 están pensados para poder ir soldando los componentes específicos y necesarios para poder arrancar el micro controlador, en cada paso se explicaran la utilidad que tiene cada componente (resistencias, Condensadores eléctricos, cristal oscilador), si bien no es un taller especifico de electrónica, la idea principal es poder entender el funcionamiento básico del hardware propuesto.

Este taller inicial plantea poder ''perder el miedo'' al trabajo con electrónica, soldar y manipular componentes, por tanto la mecánica de trabajo apunta a aprender la psico-motricidad fina necesaria para poder soldar y no tanto a indagar sobre conceptos técnicos que serán abordados a lo largo de las siguientes secuencias didácticas.

\subsubsection{recursos}
El aula:
\begin{itemize}
  \item Mesa adecuada para trabajar, una mesa por cada 4 participantes.
  \item Zapatillas eléctricas, una por mesa de trabajo.
\end{itemize}
El espacio de trabajo tiene que ser amplio, ventilado y con buena iluminación para poder trabajar y poder tener espacio para manipular los soldadores de estaño.    

para el docente:
\begin{itemize}
  \item proyector
  \item computadora con sistema ICARO instalado
\end{itemize}
para los participantes:
\begin{itemize}
  \item Soldador de estaño.
  \item Estaño.
  \item Pinza cutter.
  \item Computadora con el software ICARO instalado.
  \item Cable usb impresora (con conector tipo B).
  \item Multimetro.
  \item Des soldador.
  \item Esponja humedecida para limpiar el soldador
  \item Componentes electrónicos para soldar modulo 1 y 2 placa ICARO (Anexo 2) 
\end{itemize}

\subsubsection{recomendaciones}

\begin{itemize}
  \item Preparar el laboratorio por lo menos 30 minutos antes de empezar el taller y calentar los soldadores.

  \item Limpiar la punta de los soldadores y aplicar una película de estaño (en caso de los soldadores de punta metálica).

  \item Separar los componentes electrónicos a utilizar en el taller y repartir entre las mesas de trabajo.

\end{itemize}

\subsubsection{anexos}
\begin{itemize}
  \item Anexo 2 para el armado de la placa ICARO NP07 (modulo 1 y 2).
\end{itemize}
\newpage

%\chapter*{Secuencia didáctica}
\section*{Planificación de clase Numero: 2}
\subsubsection{Temática}

\subsubsection{Tiempo}
3 Horas
\subsubsection{NAP}

(aca pongo el nap).

\subsubsection{Resumen}



\subsubsection{Objetivos}


\subsubsection{actividades}

desarrollo:

\end{itemize}

\subsubsection{metodologia}

\subsubsection{recursos}
El aula:
\begin{itemize}
  \item Mesa adecuada para trabajar, una mesa por cada 4 participantes.
  \item Zapatillas eléctricas, una por mesa de trabajo.
\end{itemize}
El espacio de trabajo tiene que ser amplio, ventilado y con buena iluminación para poder trabajar y poder tener espacio para manipular los soldadores de estaño.    

para el docente:
\begin{itemize}
  \item proyector
  \item computadora con sistema ICARO instalado
\end{itemize}
para los participantes:
\begin{itemize}
  \item Soldador de estaño.
  \item Estaño.
  \item Pinza cutter.
  \item Computadora con el software ICARO instalado.
  \item Cable usb impresora (con conector tipo B).
  \item Multimetro.
  \item Des soldador.
  \item Esponja humedecida para limpiar el soldador
  \item Componentes electrónicos para soldar modulo 1 y 2 placa ICARO (Anexo 2) 
\end{itemize}

\subsubsection{recomendaciones}

\begin{itemize}
  \item Preparar el laboratorio por lo menos 30 minutos antes de empezar el taller y calentar los soldadores.

  \item Limpiar la punta de los soldadores y aplicar una pelicula de estaño (en caso de los soldadores de punta metálica).

  \item Separar los componentes electrónicos a utilizar en el taller y repartir entre las mesas de trabajo.

\end{itemize}
%\subsubsection{matriz de evaluación}
%bla
\subsubsection{anexos}
\begin{itemize}
  \item Anexo 2 para el armado de la placa ICARO NP07 (modulo 3).
\end{itemize}

\chapter*{Planificación de clase Numero: 3}
\subsubsection{Temática}
Introducción a la robótica, armado modulo 4 de la placa ICARO

\subsubsection{Tiempo}
3 Horas

\subsubsection{Resumen}

Luego de armar el modulo 4 de la placa NP07 ICARO, se procedera trabajar los concepto de sensores digitales y analogicos, 

\subsubsection{Objetivos}
\begin{itemize}
  \item Comprender el concepto de sensores digitales y analogicos.
  \item Comprender el concepto de conversión analogica digital.
  \item Modificar el flujo de ejecución de un algoritmo mediante la estructura de control IF usando sensores fisicos.
  \item Concepto de actuadores y sensores.
\end{itemize}

\subsubsection{actividades}
inicio:
\begin{itemize}
  \item Se repartirán los componentes y herramientas para el armado del modulo 4.

\end{itemize}
desarrollo:
\begin{itemize}
  \item Se procederá al soldado de los componentes. 
  \item Luego de ensamblado el modulo 4, explicar como es el uso de los sensores digitales y analogicos.
  \item trabajar sobre algoritmos de ''toma de decisiones'' con la estructura de control IF.
\end{itemize}
\subsubsection{metodología}

La función principal del modulo 4 de la placa NP07 es para poder usar los sensores analógicos y digitales, estos sensores permiten trabajar con secuencias IF / THEN / ELSE de forma practica.

Los sensores digitales devuelven un valor de 0 o 1 en función de si fueron activados o no, sin embargo los sensores analógicos (mediante la conversión analógica digital) entregan valores discretos de 0 a 1023. Conceptualizar el uso de sensores para mover actuadores (en este caso los LEDs del puerto ''B'' como representación de un actuador).

Es importante tratar de no enseñar recetas y en su lugar, favorecer la discusión sobre nuevas formas de abordar el mismo problema, tratar de fomentar el desarrollo de nuevos algoritmos para resolver el mismo problema planteado (uso de la estructura de control IF mediante los sensores analogicos).

\subsubsection{recursos}
El aula:
\begin{itemize}
  \item Mesa adecuada para trabajar, una mesa por cada 4 participantes.
  \item Zapatillas eléctricas, una por mesa de trabajo.
\end{itemize}
El espacio de trabajo tiene que ser amplio, ventilado y con buena iluminación para poder trabajar y poder tener espacio para manipular los soldadores de estaño.    

para el docente:
\begin{itemize}
  \item proyector
  \item computadora con sistema ICARO instalado
\end{itemize}
para los participantes:
\begin{itemize}
  \item Soldador de estaño.
  \item Estaño.
  \item Pinza cutter.
  \item Computadora con el software ICARO instalado.
  \item Cable usb impresora (con conector tipo B).
  \item Multimetro.
  \item Des soldador.
  \item Esponja humedecida para limpiar el soldador
  \item Componentes electrónicos para soldar el modulo 4 de la placa ICARO (Anexo 2) 
  \item Sensores analogicos (infra rojos, cny70, LDR, microfono electrec).
  \item Sensores digitales (botones ''final de carrera'').
  
\end{itemize}

\subsubsection{recomendaciones}

\begin{itemize}
  \item Preparar el laboratorio por lo menos 30 minutos antes de empezar el taller y calentar los soldadores.

  \item Limpiar la punta de los soldadores y aplicar una película de estaño (en caso de los soldadores de punta metálica).

  \item Separar los componentes electrónicos a utilizar en el taller y repartir entre las mesas de trabajo.

\end{itemize}

\subsubsection{anexos}
\begin{itemize}
  \item Anexo 2 para el armado de la placa ICARO NP07 (modulo 4).
\end{itemize}
\newpage


\chapter*{Planificación de clase Numero: 4}
\subsubsection{Temática}
Introducción a la robótica, armado modulo 5 y 6 placa ICARO

\subsubsection{Tiempo}
3 Horas

\subsubsection{Resumen}

Con esta secuencia didáctica se termina de montar todos los componentes de la placa NP07, dejando una placa lista y funcional para trabajar con robótica.

Los módulos 5 y 6 habilitan la fuente de alimentación externa (baterias) y el puente H (L293) encargado de manejar los dos motores de corriente continua que puede administrar el hardware ICARO. 

En esta clase se busca que los participantes comprendan el manejo de un motor de corriente continua (con inversión de giro y pwm) y conceptos de electronica para comprobación de la fuente de alimentación de la placa NP07.
Para controlar el hardware de forma autónoma se usara el adaptador buetooth HC05 que permite una comunicación wireless punto a punto emulando un puerto serial (/dev/rfcom0).

\subsubsection{Objetivos}
que los participantes logren:
\begin{itemize}
  \item Comprender el trabajo de los motores de corriente continua y sus diferencias con los motores de ''paso a paso''.
  \item Comprender el trabajo de los sistemas PWM.
  \item Ley de ohm en electrónica.
  \item Trabajar los conceptos de repeticiones (FOR, WHILE) y saltos condicionales (IF, THEN, ELSE) aplicados a un robot autónomo.
\end{itemize}

\subsubsection{actividades}
inicio:
\begin{itemize}
  \item Se repartirán los componentes y herramientas para el armado del modulo 5 y 6.
\end{itemize}
desarrollo:
\begin{itemize}
  \item Se procederá al soldado de los componentes. 
  \item Luego de ensamblado el modulo 5 y 6, se repartirán motores de corriente continua y baterías para poder ensamblar un robot autónomo.
  \item Mediante un conector HC05 (bluetooth) controlar el robot mediante PYTHON.
  \item Diseñar los algoritmos para: 
  \begin{itemize}
    \item Robot evasor de obstáculos.
    \item Robot seguidor de lineas.
    \item Robot seguidor de luz.
  \end{itemize}
  
\end{itemize}

\subsubsection{metodología}

Una ves armado el modulo 5 y 6 de la placa NP07, se procederá a probar el uso de motores de corriente continua, la fuente de alimentación (usando baterías y/o una fuente externa de 9V a 12V).

Esta secuencia didáctica busca ser un integrador de las clases anteriores, permitiendo trabajar sobre actuadores como los motores de corriente continua y servo motores (a través de PWM) interactuando con sensores analógicos y digitales. 

Es importante tratar de no enseñar recetas y en su lugar, favorecer la discusión sobre nuevas formas de abordar el mismo problema y tratar de fomentar el desarrollo de nuevos algoritmos para resolver el mismo problema planteado (uso de la estructura de control IF mediante los sensores analogicos para control de motores de corriente continua).

\subsubsection{recursos}
El aula:
\begin{itemize}
  \item Mesa adecuada para trabajar, una mesa por cada 4 participantes.
  \item Zapatillas eléctricas, una por mesa de trabajo.
\end{itemize}
El espacio de trabajo tiene que ser amplio, ventilado y con buena iluminación para poder trabajar y poder tener espacio para manipular los soldadores de estaño.    

para el docente:
\begin{itemize}
  \item proyector
  \item computadora con sistema ICARO instalado
\end{itemize}
para los participantes:
\begin{itemize}
  \item Soldador de estaño.
  \item Estaño.
  \item Pinza cutter.
  \item Computadora con el software ICARO instalado.
  \item Cable usb impresora (con conector tipo B).
  \item Multimetro.
  \item Des soldador.
  \item Esponja humedecida para limpiar el soldador
  \item Componentes electrónicos para soldar el modulo 5 y 6 de la placa ICARO (Anexo 1) 
\end{itemize}

\subsubsection{recomendaciones}

\begin{itemize}
  \item Preparar el laboratorio por lo menos 30 minutos antes de empezar el taller y calentar los soldadores.

  \item Limpiar la punta de los soldadores y aplicar una película de estaño (en caso de los soldadores de punta metálica).

  \item Separar los componentes electrónicos a utilizar en el taller y repartir entre las mesas de trabajo.

\end{itemize}

\subsubsection{anexos}
\begin{itemize}
  \item Anexo 2 para el armado de la placa ICARO NP07 (modulo 5 y 6).
\end{itemize}
\newpage


\end{document}
