\chapter*{Planificación de clase Numero: 3}
\subsubsection{Temática}
Introducción a la robótica, armado modulo 4 de la placa ICARO

\subsubsection{Tiempo}
3 Horas

\subsubsection{Resumen}

Luego de armar el modulo 4 de la placa NP07 ICARO, se procedera trabajar los concepto de sensores digitales y analogicos, 

\subsubsection{Objetivos}
\begin{itemize}
  \item Comprender el concepto de sensores digitales y analogicos.
  \item Comprender el concepto de conversión analogica digital.
  \item Modificar el flujo de ejecución de un algoritmo mediante la estructura de control IF usando sensores fisicos.
  \item Concepto de actuadores y sensores.
\end{itemize}

\subsubsection{actividades}
inicio:
\begin{itemize}
  \item Se repartirán los componentes y herramientas para el armado del modulo 4.

\end{itemize}
desarrollo:
\begin{itemize}
  \item Se procederá al soldado de los componentes. 
  \item Luego de ensamblado el modulo 4, explicar como es el uso de los sensores digitales y analogicos.
  \item trabajar sobre algoritmos de ''toma de decisiones'' con la estructura de control IF.
\end{itemize}
\subsubsection{metodología}

La función principal del modulo 4 de la placa NP07 es para poder usar los sensores analógicos y digitales, estos sensores permiten trabajar con secuencias IF / THEN / ELSE de forma practica.

Los sensores digitales devuelven un valor de 0 o 1 en función de si fueron activados o no, sin embargo los sensores analógicos (mediante la conversión analógica digital) entregan valores discretos de 0 a 1023. Conceptualizar el uso de sensores para mover actuadores (en este caso los LEDs del puerto ''B'' como representación de un actuador).

Es importante tratar de no enseñar recetas y en su lugar, favorecer la discusión sobre nuevas formas de abordar el mismo problema, tratar de fomentar el desarrollo de nuevos algoritmos para resolver el mismo problema planteado (uso de la estructura de control IF mediante los sensores analogicos).

\subsubsection{recursos}
El aula:
\begin{itemize}
  \item Mesa adecuada para trabajar, una mesa por cada 4 participantes.
  \item Zapatillas eléctricas, una por mesa de trabajo.
\end{itemize}
El espacio de trabajo tiene que ser amplio, ventilado y con buena iluminación para poder trabajar y poder tener espacio para manipular los soldadores de estaño.    

para el docente:
\begin{itemize}
  \item proyector
  \item computadora con sistema ICARO instalado
\end{itemize}
para los participantes:
\begin{itemize}
  \item Soldador de estaño.
  \item Estaño.
  \item Pinza cutter.
  \item Computadora con el software ICARO instalado.
  \item Cable usb impresora (con conector tipo B).
  \item Multimetro.
  \item Des soldador.
  \item Esponja humedecida para limpiar el soldador
  \item Componentes electrónicos para soldar el modulo 4 de la placa ICARO (Anexo 2) 
  \item Sensores analogicos (infra rojos, cny70, LDR, microfono electrec).
  \item Sensores digitales (botones ''final de carrera'').
  
\end{itemize}

\subsubsection{recomendaciones}

\begin{itemize}
  \item Preparar el laboratorio por lo menos 30 minutos antes de empezar el taller y calentar los soldadores.

  \item Limpiar la punta de los soldadores y aplicar una película de estaño (en caso de los soldadores de punta metálica).

  \item Separar los componentes electrónicos a utilizar en el taller y repartir entre las mesas de trabajo.

\end{itemize}

\subsubsection{anexos}
\begin{itemize}
  \item Anexo 2 para el armado de la placa ICARO NP07 (modulo 4).
\end{itemize}
\newpage
