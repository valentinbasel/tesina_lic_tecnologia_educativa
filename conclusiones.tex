\chapter{Reflexiones finales}
%\section{Introducción}

%En este capitulo se tratara de dar cuenta sobre las consideraciones necesarias para tener en en cuenta, a la hora de poder llevar ''a buen puerto'' un proyecto que implique trabajar con sotware y hardware libre para su implementación como herramienta de uso pedagogico.
Como parte de una consideración final, podemos decir que todo proyecto de transferencia, requiere capacitación y desarrollo y, a su vez, como menciona \cite{bareno2011metodologia} hay siete pasos necesarios para la adopción de un proyecto de transferencia tecnológica (elección, adquisición, adopción, absorción, aplicación, difusión y desarrollo) que son necesarios tener en cuenta  a la hora de poder trabajar con un proyecto de transferencia de tecnología. Si bien específicamente Camargo Bareño se refiere a un proyecto de transferencia de tecnología para uso industrial, da cuenta de las complejidades inherentes al desarrollo de un proyecto que implique el uso de tecnologías de software y hardware libre.
 
Por lo tanto, hay que tener en cuenta esta gran complejidad a la hora de tomar la decisión de migrar, o mas bien adoptar, este proyecto de transferencia de tecnología educativa abierta, porque si bien es un proyecto para implementar un recurso educativo abierto para la enseñanza de lenguajes de programación, en ultima instancia este proyecto implica una discusión sobre la soberanía tecnológica en los colegios.

 \section{Robótica como herramienta construccionista}
 
 La robótica como herramienta para la enseñanza debe ser tomada desde la perspectiva construccionista \citep{pitti_experiencias_2010} y usada como instrumento transversal en el proceso educativo \citep{seymour_papert_maquina_1995}, dado que las múltiples disciplinas que intervienen en el desarrollo de un robot, permiten abordar un gran numero de conceptos de forma practica.
 
 Sin embargo, por la enorme complejidad de la la robótica \citep{reyes_cortes_robotica:_2011} como  sub rama de la carrera de mecatronica, no debería ser considerada una rama de estudio propia para los colegios secundarios, si no como herramienta de trabajo suplementario para el aprendizaje de otras asignaturas. Concretamente, la utilización de mecanismo de automatización o robotica, permite trabajar conceptos complejos de las ciencias computacionales mediante un abordaje practico \citep{sanchez_robotica_2012}.
 
 %Por lo tanto, pensar en la robotica
 
 
 \section{Programa o seras programado}
 
 La enseñanza de programación no solo es importante en el marco de la actual ley Nacional de Educación N\grad 26.206 si no que también es reconocida como una necesidad a nivel de capacitación laboral, lo que se ha vuelto un requerimiento para los colegios especializados en ''orientación informática'' lograr tener un espacio curricular capas de estar actualizado y generar conocimientos que sigan siendo útiles a los discentes a pesar del paso del tiempo.
 
 Sin embargo, la enseñanza de un lenguaje de programación no debería ser solo un requerimiento industrial, donde los colegios funcionen como capacitadores de ''mano de obra'' especializada para las empresas del sector informático, si no poder preparar individuos de pensamiento crítico, capaces de comprender no solamente las características técnicas de la tecnología digital si no también las implicaciones sociales y políticas del uso de las mismas. En la actualidad, el uso de sistemas digitales programables (micro-procesadores, micro-controladores) es cada vez común y la necesidad de poder contar con desarrolladores para esos sistemas es fundamental pero, ademas de la apropiación de los discentes del pensamiento algorítmico, es necesario que la escuela sea un lugar de reflexión sobre las implicaciones sociales del uso y desarrollo de software.
 Podríamos decir que en el contexto actual, en una sociedad cada ves mas atravesada por las tecnologías digitales, sistemas de inteligencia artificial y practicas monopólicas con respecto al licenciamiento de software, o ''programas o seras programado'' \citep{rushkoff2010program}.
 
 \section{Lo instrumental también es político}
 
 La apropiación (o imposición) de una herramienta (física o virtual) modifica el entorno donde nos desenvolvemos diariamente  y ,por ende, nuestra concepción de las relaciones de poder y dominación, por lo tanto podemos entender a la tecnología como una herramienta que posibilita una forma de liberación o colonización sobre nuestros derechos individuales y colectivos, no solamente una solución ''técnica'' a un problema puntual.
 
 El movimiento feminista acuño en los setenta el lema ''lo personal es político'' \citep{hanisch1969personal}, como referencia a que el espació de las esferas llamadas ''privadas'' (el hogar, la familia, el propio cuerpo) era también escenario de dominación y por tanto lugar de luchas emancipadoras. En ese contexto, los movimientos de usuarios y desarrolladores de software libre, asumen que el uso y la adopción de una herramienta de software, también conlleva una postura ideológica al respecto. Desde esa perspectiva, la adopción de por parte del estado de herramientas libres, y sobre todo en las instituciones educativas, debería ser parte fundamental a la hora de elegir una tecnología en particular para implementar en el aula.
 
 Sin embargo, la escasa documentación, falta de cursos de capacitación específicos y en muchos casos la falta de herramientas diseñadas específicamente para el uso en educación, hace que la adopción, por parte de las instituciones, de un recurso educativo abierto (REA) sea un tema complejo y en muchos casos casi imposible. Ademas, existe  un  número  muy reducido  de investigaciones  sobre  el desarrollo curricular especifico a la robótica como herramienta educativa, y concretamente  como herramienta para la enseñanza de lenguajes de programación.  
 
 Este proyecto de transferencia de tecnología educativa da cuenta de las necesidades y complejidades inherentes a la implementación, adopción y capacitación del cuerpo docente de la institución que decida adoptar un proyecto de transferencia como este, tratando de ''armonizar'' las necesidades de la escuela y las propuestas esgrimidas por las comunidades de desarrolladores de software libre.
 
 
